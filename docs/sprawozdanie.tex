\documentclass[a4 122pt]{article}

\usepackage{color}
\usepackage{polski}
\usepackage[utf8]{inputenc}
\usepackage{amsmath}
\usepackage{amsthm}
\usepackage{todonotes}
\usepackage{geometry}
\usepackage{hyperref}

\geometry{
  body={6.5in, 8.5in},
  left=0.7in,
  top=0.8in,
  bottom=0.8in
}

\title{Projekt i realizacja agenta do okresowego monitorowania treści wskazanych stron internetowych i raportowania zmian treści na tych stronach.}
\author{Projekt na przedmiot \textbf{ISI}\\Michał Kaszlej}



\begin{document}

\maketitle

	\section{Wybór dziedziny}

		\subsection{Motywacja}
		
			Przystępując do implementacji agenta, chciałem tak wyznaczyć dziedzinę jego działania aby powstała aplikacja była czymś więcej niż tylko projektem na zaliczenie.
			
			Takie założenie skierowało moją uwagę na problem który w ostatnim czasie stanął przed grupą znajomych socjologów zaangażowanych w tworzenie projektu 
			\href{http://ozkultura.pl}{Obserwatorium żywej kultury} nazywanego też dalej OŻK.

		\subsection{Obserwatorium żywej kultury}
		
			Obserwatorium żywej kultury jest szeroko zakrojonym projektem skoncentrowanym na zbieraniu i gromadzeniu wiedzy i informacji pomocnych w badaniach socjologicznych.
			
			Jednym z głównych składników projektu jest tworzenie ogólnopolskiej mapy przejawów \href{http://ozkultura.pl/node/111}{żywej kultury}:
			
			\begin{quote}
				\textit{Żywa kultura to wielowymiarowe środowisko (milieu) życia jednostek i grup społecznych oraz funkcjonowania instytucji społecznych, w którym zachodzą dynamiczne procesy, rozwijają się praktyki kulturowe, powstają  mniej lub bardziej trwałe rezultaty (materialne i niematerialne wytwory) praktyk. Zarówno jednostki, grupy, instytucje, procesy, praktyki, jak i ich wytwory charakteryzują się zróżnicowanym, najczęściej wielowarstwowym i zmiennym  nacechowaniem aksjologicznym oraz zróżnicowanymi, zmiennymi i wielowarstwowymi, najczęściej polisemicznymi, znaczeniami.}
			\end{quote}
			
			W takim rozumieniu każdy sklep mięsny, czy basen jest przejawem żywej kultury i informacja o jego istnieniu powinna zostać uwzględniona w Obserwatorium żywej kultury.

		\subsection{Przyjęte metody pozyskiwania danych do OŻK}

			Dane pozyskiwane są przez uczestników projektu poprzez przeszukiwanie internetu i wypełnianie odpowiednich pól w arkuszu kalkulacyjnym.
			
			Poniżej znajduje się wyciąg z instrukcji dotyczącej pozyskiwania danych dla OŻK:
			
			
			\begin{quote}

				\textit{5) Danych poszukujemy w następujący sposób:\\
				Generalna uwaga: przeszukujemy strony sięgając najdalej do 10 pierwszych adresów;}
				
				
				\begin{itemize}
				
					\item \textit{punktem wyjścia są oficjalne strony GMIN, oficjalne strony MIAST i POWIATÓW oraz LINKI do nich opisujące kategorie infrastrukturalne wymienione w tesaurusie;}
					\item \textit{niektóre gminy i powiaty mają bardzo dobre strony w wikipedii;}
					\item \textit{warto sprawdzić informacje z BIP, ale to raczej dla porządku}
					\item \textit{trzeba uważać i sprawdzać – np. na stronie powiatu jest link do sal bankietowych w powiecie; wchodzimy: sale są podane tylko w miejscowościach, bez zaznaczenia jaka to gmina – trzeba oczywiście sprawdzić gminę!}
					\item \textit{dla wielu prywatnych kategorii infrastrukturalnych może być pomocna PANORAMA FIRM i lub REGON}
					\item \textit{warto przejrzeć też strony i portale tematyczne: informacji kulturalnych dla województwa i powiatów, chórtownia, regiopedia, kultura ludowa itp.}
				\end{itemize}
				
				\textit{6) UWAGA : cały sens tego badania polega na STARANNOŚCI.}
			\end{quote}
	
			Ponadto każdy uczestnik odpowiada wyłącznie za ograniczony i określony zbiór gmin.
			Poniżej zamieszczam wykaz gmin przydzielonych mojej znajomej:
			
			\vspace{0.5cm}
			\begin{center}
			\begin{tabular}{cccc}
				\centering
				
				\textbf{Miasta} & \textbf{Gminy miejske} & \textbf{Gminy miejsko-wiejskie} & \textbf{Gminy wiejskie} \\ \hline
				Góra Kalwaria		 	& 	Pionki 	& Iłża 				& Gózd \\\hline
				Konstancin-Jeziorna 	&	Sierpc	& Skaryszew			& Jastrzębia\\\hline
				Piaseczno				&			& Góra Kalwaria		& Jedlińsk\\\hline
				Tarczyn					&			& Konstancin-Jeziorna & Jedlnia-Letnisko\\\hline
				Pionki					&			& Piaseczno				& Kowala\\\hline
				Iłża					&			& Tarczyn			&	Pionki	\\\hline
				Skaryszew				&			&	Lesznowola		&	Przytyk \\\hline
				Prażmów					&			&					& Wierzbica\\\hline
																	& & & Wolanów\\ \hline
																	& & & Zakrzew\\\hline
																	& & & Gozdowo\\\hline
																	& & & Mochowo\\\hline
																	& & & Rościszewo\\\hline
																	& & & Sierpc\\\hline
																	& & & Szczutowo\\\hline
																	& & & Zawidz\\
			\end{tabular}
			\end{center}
	
		\subsection{Zbierane dane}

			Zbieranie danych polega na wypełnianiu oddzielnego akrusza dla każdej gminy.
			Każdy arkusz zawiera następujące kolumny:
			
			\begin{enumerate}
				\item{Nazwa gminy lub miasta na prawach powiatu}
				\item{Numer TERYT}
				
					Każdej gminie w Polsce przypisany jest numer TERYT. 
					
				\item{Indeks kategorii}
				
					Przypisanie obiektu do jednej z 28 kategorii, według \href{http://ozkultura.pl/node/1800}{zamieszczonej na stronie tabeli}.
					
				\item{Kategoria własności obiektu}
					
					Możliwe wartości:
					\begin{itemize}
						\item P = prywatna
						\item S = samorządowa
						\item PA = państwowa (budżetowa)
						\item N = NGO's (organizacje pozarządowe)
						\item I = inne (np. mieszany typ, współprowadzenie itd.)
					\end{itemize}
					
				\item{Nazwa obiektu}
				\item{Adres obiektu}
				\item{Telefon/Email}
				\item{Liczba miejsc/Zatrudnienie}
				\item{Liczba funkcji obiektu}
				
					Cytat z instrukcji wypełniania:
					
					\begin{quote}
					\textit{Wpisujemy symbol 1 = instytucja lub organizacja ma tylko 1 funkcję podstawową ( np.: jest biblioteką , nieważne, że odbywają się w niej różne zajęcia); symbol 2 = pod jedną nazwą i adresem realizowane są 2 lub więcej funkcji (np.: w Centrum Kultury jest Kino wymienione z nazwy, a nie napisano, że Centrum ma salę kinową);}
					\end{quote}
				
				\item{PRZYPADKI WĄTPLIWE}
				
					Możliwe wartości:
					\begin{itemize}
						\item I = brak przypisania do kategorii
						\item FW = brak informacji na temat formy własności
						\item BD = brak części danych
						\item NA = wątpliwość dotycząca aktualności danych
					\end{itemize}
					
			\end{enumerate}
	
		\subsection{Ocena przyjętego w OŻK sposobu pozyskiwania danych}
		
			Ogrom pracy  stojący nad członkiem projektu OŻK jest niewyobrażalny.
			Stoi on przed zadaniem zindeksowania wszelkich przejawów kultury na ogromnym obszarze.
			
			Jego zadaniem jest przeglądanie internetu w poszukiwaniu pływalni, kościołów, sklepów spożywczych itp i skrzętne wprowadzanie ich do arkusza.
			
			Jest to praca zakrojona na tygodnie, jeśli nie miesiące.
			
			Wydaje mi się, że znacznie lepszy efekt uzyskać można poprzez zastosowanie agenta i poświęcenie zarezerwowanego czasu na weryfikację poprawności wyników. 
			
			Aby odzyskać zapracowanych znajomych, zdecydowałem się więc na takie właśnie określenie dziedziny implementowanego agenta. 
	
	\section{Źródła pozyskiwania danych}
		
			Realizację projektu rozpocząłem od zbadania możliwych źródeł pozyskiwania danych.
		
			Założyłem, że rozwiązaniem optymlanym byłoby uzyskanie możliwie dużej liczby danych przy możliwie niewielkiej liczbie źródeł.
			Założenie takie wynika z faktu, że im więcej źródeł, tym większy stopień komplikacji agenta.
			
			\subsection{KRS}
			
				Na stronie Ministerstwa sprawiedliwości znajduje się \href{https://ems.ms.gov.pl/krs/wyszukiwaniepodmiotu?t:lb=t}{wyszukiwarka numerów Krajowego Rejestru Sądowego.}
		
			\subsection{BiP}
			
			\subsection{Wikipedia}
			
			\subsection{Google}
			
			\subsection{zumi}
			
	\section{Koncepcja projektu}
	
		Niezbędne jest stworzenie programu który
		
	\section{Przetwarzanie danych}
	
	\section{Realizacja projektu}
	
	\section{Możliwości rozszerzenia agenta}
\end{document}	
